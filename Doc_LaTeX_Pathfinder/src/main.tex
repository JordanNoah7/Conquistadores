%! Author = dyfmeks
%! Date = 20/05/2024

% Preamble
\documentclass[stu, 12pt, letterpaper, donotrepeattitle, floatsintext, natbib]{apa7}

% Packages
\usepackage[utf8]{inputenc}
\usepackage{comment}
\usepackage{marvosym}
\usepackage{graphicx}
\usepackage{float}
\usepackage[normalem]{ulem}
\usepackage[spanish]{babel}
\usepackage[style=apa,sortcites=true,sorting=nyt,backend=biber]{biblatex}
\usepackage{standalone}
\selectlanguage{spanish}
\useunder{\uline}{\ul}{}
\newcommand{\myparagraph}[i]{\paragraph{#1}\mbox{}\\}
%\usepackage[section]{placeins}
%\usepackage{csquotes}
%\usepackage{array,fancyvrb,graphicx,verbatim,xurl}

% Config
%\DeclareLanguageMapping{american}{american-apa}
%\addbibresource{bibliography.bib}
\thispagestyle{empty}
\title{\large DISE\~{N}O Y DESARROLLO DE UN SISTEMA DE INFORMACI\'ON INTEGRAL PARA LA GESTI\'ON DE CONQUISTADORES, ACTIVIDADES Y
PROGRESOS EN EL CLUB DE CONQUISTADORES}
\shorttitle{Sistema Integral para la Gesti\'on del Club de Conquistadores}
\author{John Jordan Quispe Supo}
\affiliation{Instituto de Educaci\'on Superior Privado del Sur}
\course{Desarrollo de Sistemas de Informaci\'on}
\professor{Gustavo Delgado Ugarte}
\duedate{20/05/2024}

%\abstract{Este proyecto se centra en el dise\~{n}o y desarrollo de un Sistema de Informaci\'on Integral para la Gesti\'on de
%Conquistadores, Actividades y Progresos en el Club de Conquistadores. El sistema propuesto permitir\'a el registro de
%nuevos conquistadores, la planificaci\'on y registro de actividades, el seguimiento del avance de los conquistadores en
%sus especialidades y clases, y la toma de asistencia y actividades semanales en la unidad. Este trabajo busca optimizar
%la gesti\'on del club, mejorar la eficiencia de las actividades y proporcionar una plataforma web para el seguimiento y
%desarrollo de los conquistadores. Se espera que este sistema integral mejore significativamente la administraci\'on del
%club y enriquezca la experciencia de los conquistadores.}
%\keywords{Sistema de Informaci\'on Integral, Gesti\'on de Conquistadores, Registro de Actividades, Seguimiento de Progresos,
%Club de Conquistadores, Planificaci\'on de Actividades, Toma de Asistencia, Actividades Semanales, Especialidades y Clases
%de Conquistadores, Plataforma Web, Optimizaci\'on de la Gesti\'on, Eficiencia de las Actividades}

% Document
\begin{document}
    \maketitle

%   Indices
    \pagenumbering{roman}
%   Contenido
    \renewcommand\contentsname{\large\'Indice}
    \tableofcontents
    \setcounter{tocdepth}{2}
    \newpage
%   Figuras
    \renewcommand\listfigurename{\large\'Indice de figuras}
    \listoffigures
    \newpage
%   Tablas
    \renewcommand\listtablename{\large\'Indice de tablas}
    \listoftables
    \newpage

%   Cuerpo
    \pagenumbering{arabic}


    \section{\large T\'ITULO}
    \noindent DISE\~{N}O Y DESARROLLO DE UN SISTEMA DE INFORMACI\'ON INTEGRAL PARA LA GESTI\'ON DE CONQUISTADORES, ACTIVIDADES Y
    PROGRESOS EN EL CLUB DE CONQUISTADORES


    \section{\large DATOS DEL AUTOR}
    \noindent
    \begin{tabular}{ll}
        \textbf{Nombres}    & : John Jordan                                                             \\
        \textbf{Apellidos}  & : Quispe Supo                                                             \\
        \textbf{Carrera}    & : Desarrollo de Sistemas de Informaci\'on                                 \\
        \textbf{Rese\~{n}a} & : Soy egresado del Instituto del Sur y actualmente me encuentro laborando \\ & en Soluciones de Informaci\'on NextSoft S.A.C.
    \end{tabular}


    \section{\large RESUMEN}
    Este proyecto se centra en el dise\~{n}o y desarrollo de un Sistema de Informaci\'on Integral para la Gesti\'on de
    Conquistadores, Actividades y Progresos en el Club de Conquistadores. El sistema propuesto permitir\'a el registro de
    nuevos conquistadores, la planificaci\'on y registro de actividades, el seguimiento del avance de los conquistadores en
    sus especialidades y clases, y la toma de asistencia y actividades semanales en la unidad. Este trabajo busca optimizar
    la gesti\'on del club, mejorar la eficiencia de las actividades y proporcionar una plataforma web para el seguimiento y
    desarrollo de los conquistadores. Se espera que este sistema integral mejore significativamente la administraci\'on del
    club y enriquezca la experciencia de los conquistadores.


    \section{\large INTRODUCCI\'ON}

    \subsection{Descripci\'on de la Unidad de Estudio}
    \noindent
    \begin{tabular}{ll}
        \textbf{RUC}                            & : 20202020202                                        \\
        \textbf{Raz\'on Social}                 & : Club de Conquistadores "Las \'Aguilas"             \\
        \textbf{Fecha de Inicio de Actividades} & : 20/05/2012                                         \\
        \textbf{Actividad principal}            & : Ayudar a los adolescentes a crecer en las 3 partes \\ & esenciales que son fisico mental y espiritual
    \end{tabular}

    \subsection{titulo 2}


    \section{titulo 1}

    \subsection{titulo 2}

    \subsubsection{titulo 3}

%
%    \section{Resultados}
%    \begin{table}[h]
%        \caption{Sample Basic Table}
%        \label{tab:BasicTable}
%        \begin{tabular}{@{}llr@{}}
%            \toprule
%            \multicolumn{2}{c}{Item} \\ \cmidrule(r){1-2}
%            Animal    & Description & Price \\ \midrule
%            Gnat      & per gram    & 13.65 \\
%            & each        & 0.01  \\
%            Gnu       & stuffed     & 92.50 \\
%            Emu       & stuffed     & 33.33 \\
%            Armadillo & frozen      & 8.99  \\ \bottomrule
%        \end{tabular}
%    \end{table}
%    \begin{figure}
%        \caption{Descripci\'on de figuras.}
%        %\includegraphics[scale=]{Figure1.pdf}
%        \label{fig:Figure1}
%    \end{figure}
%
%
%    \section{Discusi\'on}
%
%
%    \section{Conclusiones}
%    \printbibliography
%    \appendix
%
%
%    \section{Instrumentos}
%    \label{Ap. instrumentos}
%    ver detalles en las figuras
%    In this picture you can see a bar graph that shows. In this picture you can see a bar graph that shows
%
%
%    \section{Datos}
%    \label{Ap.Datos}
%    detalles en las tablas
\end{document}
